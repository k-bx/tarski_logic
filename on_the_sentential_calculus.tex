\section{On the sentential calculus}
\begin{enumerate}
\item Give examples of specifically mathematical expressions from the
  fields of arithmetic and geometry.
  \begin{itemize}
  \item \textbf{Arithmetic:}
    $$3 \div 4$$
    $$21 + 42 \times 3$$
    $$2^8 - 2^6$$
  \item \textbf{Geometry:}
    $$\text{The volume of a sphere of radius 3}$$
    $$\text{The sum of the measures of angles in any triangle}$$
    $$\text{The ratio of a circle's circumference to its diameter}$$
  \end{itemize}
\item Differentiate in the following two sentences between the
  specifically mathematical expressions and those belonging to the
  domain of logic:
  \begin{enumerate}
    \item \emph{for any numbers x and y, if $x>0$ and $y<0$, then
      there is a number z such that $z<0$ and $x=y \cdot z$;}
      \begin{itemize}
      \item \textbf{Mathematical:} $$x>0$$ $$y<0$$ $$z<0$$ $$x=y \cdot
        z$$
      \item \textbf{Logical:} Universal quantification over $x$ and
        $y$, implication (if{\ldots}then), conjuction (and), and
        existential quantification for some $z$.
      \end{itemize}
    \item \emph{for any points A and B there is a point C which lies
      between A and B and is the same distance from A as from B.}
      \begin{itemize}
      \item \textbf{Mathematical:} points, line, distance
      \item \textbf{Logical:} Universal quantification over $A$ and
        $B$, existential quantification over $C$, $and$ operator
      \end{itemize}
  \end{enumerate}
\item Form the conjuction of the negations of the following sentential
  functions:
  $$x<3$$
  and
  $$x>3.$$
  What number satisfies this conjuction?
  \begin{itemize}
  \item $(\sim x<3) \wedge (\sim x>3)$
  \item $(x \ge 3) \wedge (x \le 3)$
  \item $x = 3$
  \end{itemize}
\item In which of its two meanings does the word ``\emph{or}'' occur in the
  following sentences:
  \begin{enumerate}
  \item \emph{two ways were open to him: to betray his country or to
    die;}
    \begin{itemize}
    \item Exclusive
    \end{itemize}
  \item \emph{if I earn a lot of money or win the sweepstake, I shall
    go on a long journey.}
    \begin{itemize}
    \item Non-exclusive
    \end{itemize}
  \end{enumerate}
  Give further examples in which the word ``\emph{or}'' is used in its first
  or in its second meaning.
  \begin{itemize}
  \item \textbf{Exclusive:}
    \begin{itemize}
    \item \emph{The package will arrive Thursday or Friday.}
    \item \emph{Jefferson or Adams will be elected president.}
    \end{itemize}
  \item \textbf{Non-exclusive:}
    \begin{itemize}
    \item \emph{We have Coke or Pepsi.}
    \item \emph{It is hot outside, or the air conditioning is broken.}
    \end{itemize}
  \end{itemize}
\item Consider the following conditional sentences:
  \begin{enumerate}
  \item \emph{if today is Monday, then tomorrow is Tuesday;}
    \begin{itemize}
    \item If it is any day except Monday, the antecedent is false, so
      the entire sentence is true.  On Monday, the consequent is true.
      Therefore, the statement is always true.
    \end{itemize}
  \item \emph{if today is Monday, then tomorrow is Saturday;}
    \begin{itemize}
    \item If it is not Monday, the sentence is true.  When it is
      Monday, the consequent, and the entire sentence is false.
    \end{itemize}
  \item \emph{if today is Monday, then the 25th of December is
    Christmas day;}
    \begin{itemize}
    \item True, since the consequent is always true.
    \end{itemize}
  \item \emph{if wishes were horses, beggars could ride;}
    \begin{itemize}
    \item Wishes are not horses, so since the antecedent is false, the
      entire statement is always true.  This case is somewhat
      ambiguous because of the use of ordinary language.
    \end{itemize}
  \item \emph{if a number x is divisible by 2 and by 6, then it is
    divisible by 12;}
    \begin{itemize}
    \item False, since 18 is divisible by both 2 and 6, but is not
      divisible by 12.
    \end{itemize}
  \item \emph{if 18 is divisible by 3 and by 4, then 18 is divisible
    by 6;}
    \begin{itemize}
    \item True, since 18 is divisible by 6.  The truth of the
      antecedent is irrelevent, since the consequent is always true.
    \end{itemize}
  \end{enumerate}
  Which of the above implications are true and which are false from
  the point of view of mathematical logic?  In which cases does the
  question of meaningfulness and of truth or falsity raise any doubts
  from the standpoint of ordinary language?  Direct special attention
  to the sentence (b) and examine the question of its truth as
  dependent on the day of the week on which it was uttered.
\item Put the following theorems into the form of ordinary conditional
  sentences:
  \begin{enumerate}
  \item For a triangle to be equilateral, it is sufficient that the
    angles of the triangle be congruent;
    \begin{itemize}
    \item If the angles of a triangle are congruent, the triangle is
      equilateral.
    \item Paraphrased as: Congruent angles on a triangle mean it is
      equilateral.
    \end{itemize}
  \item the condition: x is divisible by 3, is necessary for x to be
    divisible by 6.
    \begin{itemize}
    \item If x is divisible by 6, then x is divisible by 3.
    \item Paraphrased as: x is divisible by 3, when x is divisible by 6.
    \end{itemize}
  \end{enumerate}
  Give further paraphrases of these two sentences.
\item Is the condition: $$x{\cdot}y > 4$$ necessary or sufficient for
  the validity of: $$x > 2 \quad \text{\emph{and}} \quad y > 2 \quad
  \text{?}$$
  \begin{itemize}
  \item $x{\cdot}y > 4$ is not a sufficient condition, since $1{\cdot}5 >
    4$, but $x \ngtr 2$.
  \item It is necessary, since $x$ and $y$ must both be greater than 2,
    and $2{\cdot}2=4$.  So, $x{\cdot}y > 4$ given the lower bounds on
    $x$ and $y$.
  \end{itemize}
\item Give alternative formulations for the following sentences:
  \begin{itemize}
  \item \emph{x is divisible by 10 if, and only if, x is divisible
      both by 2 and by 5;}
    \begin{itemize}
    \item \emph{x being divisible by 10 is a necessary and sufficient
        condition for x being divisible by both 2 and by 5.}
    \end{itemize}
  \item \emph{for a quadrangle to be a parallelogram it is necessary
      and sufficient that the point of intersection of its diagonals
      be at the same time the midpoint of each diagonal.}
    \begin{itemize}
    \item \emph{a quadrangle is a parallelogram if, and only if, the
        point of intersection of its diagonals be at the same time the
        midpoint of each diagonal.}
    \end{itemize}
  \end{itemize}
  Give futher examples of theorems from the fields of arithmetic and
  geometry that have the form of equivalences.
  \begin{itemize}
  \item \textbf{Arithmetic:} $x{\cdot}y=z$ \emph{if, and only if,}
    $y{\cdot}x=z$. \text{(Symmetry)}
  \item \textbf{Geometry:} \emph{A triangle is a right triangle if,
      and only if, the square of the length of the hypotenuse is equal
      to the sum of the squares of the lengths of the other two
      sides.} (Pythagorean Theorem)
  \end{itemize}
\item Which of the following sentences are true:
  \begin{enumerate}
  \item \emph{a triangle is isosceles if, and only if, all the
      altitudes of the triangle are congruent;}
    \begin{itemize}
    \item False.  If all altitudes are congruent, the triangle must be
      equilateral.
    \end{itemize}
  \item \emph{the fact that $x \neq 0$ is necessary and sufficient for
      $x^2$ to be a positive number;}
    \begin{itemize}
    \item True.  $x^2$ is always positive, unless $x$ is 0.  $x$ being
      either positive or negative also implies that $x^2 > 0$.
    \end{itemize}
  \item \emph{the fact that a quadrangle is a square implies that all
      its angles are right angles, and conversely;}
    \begin{itemize}
    \item True, by definition.  The converse is false, an oblong
      rectangle has all right angles but is not a square.
    \end{itemize}
  \item \emph{for x to be divisible by 8 it is necessary and
      sufficient that x be divisible both by 4 and by 2} \quad ?
    \begin{itemize}
    \item False.  Take $x=12$.  $x$ is divisible by 4 and 2, but
      not by 8.
    \end{itemize}
  \end{enumerate}
\item Assuming the terms ``\emph{natural number}'' and ``\emph{product}'' (or
  ``\emph{quotient}'', respectively) to be known already, construct the
  definition of the term ``\emph{divisible}'', giving it the form of an
  equivalence.

  Likewise formulate the definition of the term ``\emph{parallel}'';
  what terms (from the domain of geometry) have to be presupposed for
  this purpose?
  \begin{itemize}
  \item \textbf{Divisible:} We say that x is divisible by y, if and
    only if, there exists some natural number z, such that the product
    of y and z is x.
  \item \textbf{Parallel:} We say that two lines, a and b, in
    Euclidean space are parallel, if and only if, every point on line
    a has the same minimum distance to a point on line b.
    \begin{itemize}
      \item Presupposes the terms ``\emph{line}'', ``\emph{Euclidean space}'', and
        ``\emph{minimum distance}''.
    \end{itemize}
  \end{itemize}
\item Translate the following symbolic expressions into ordinary language:
  \begin{enumerate}
  \item $[(\sim p) \rightarrow p ] \rightarrow p$

    If $p$ is false implies $p$, then $p$ is implied.
  \item $[(\sim p) \vee q ] \leftrightarrow (p \rightarrow q)$

    For any $p$ and $q$, not $p$ or $q$ is equivalent to $p$ implies $q$.
  \item $[\sim (p \vee q)] \leftrightarrow (p \rightarrow q)$

    For any $p$ and $q$, the negation of $p$ or $q$ is equivalent to $p$ implies $q$.
  \item $(\sim p) \vee [ q \leftrightarrow (p \rightarrow q)]$

    The negation of $p$, or whether $q$ is equivalent to $p$ implies $q$.
  \end{enumerate}
\item Formulate the following expressions in logical symbolism:
  \begin{enumerate}
    \item \emph{if not p or not q, then it is not the case that p or q.}
      
      $[(\sim p) \vee (\sim q)] \rightarrow [\sim (p \vee q)]$
    \item \emph{if p implies that q implies r, then p and q together imply r.}

      $[p \rightarrow (q \rightarrow r)] \rightarrow [(p \wedge q) \rightarrow r]$
    \item \emph{if r follows from p and if r follows from q, then r follows from p or q.}

      $[(p \rightarrow r) \wedge (q \rightarrow r)] \rightarrow [(p \vee q) \rightarrow r]$
  \end{enumerate}
\end{enumerate}
